\documentclass{article}
\usepackage[margin=1.25in]{geometry}
\usepackage{amssymb}
\usepackage{amsmath}
\usepackage{mathtools}
\usepackage[ruled, vlined, linesnumbered ]{algorithm2e}

\title{Graph Theory: Homework 1}
\author{Nikzad Khani}

\begin{document}
\maketitle

\textbf{Problem 1}

\begin{equation}
    MM^T = A + D_V
\end{equation}
where $D_V$ is a diagonal matrix where the degree of the vertex defined by
row $i$ in $M$ is the $(i, i)$ entry in $D_V$.\\

\quad Let $M$ be a $q$ by $r$ size matrix and let $A$ be a $q$ by $q$ size matrix.
Let $a_{i, j}$ be the entry in $A$ at row $i$ and column $j$. This entry
is defined by $M_iM^T_{j}$, where $M_i$ is the row vector at $i$ and $M^T_j$ is
a column vector in $M^T$. This product is equivalent to $M_i \cdot M_j$ since
a column vector at $j$ in $M^T$ is equivalent to a row vector at $j$ in $M$.
Thus, we can define $a_{i, j}$ as follows where $n$ is the length of the row
vector.

\begin{equation}
    \forall \underset{i\neq j}{i,j} \in[1, q]\quad
    a_{i, j} = M_i \cdot M_j = \sum_{t = 1}^{r}m_{i, t}m_{j, t}
\end{equation}

By definition of an incidence matrix, we know that each entry in $M$ will be a
$1$ if the vertex defined by the entry's row number is incident to the edge
defined by the entry's column number and a $0$ otherwise. When $a_{i, j}$ is
not an entry along the diagonal of $A$, there are two possible products.
The first product is when both $m_{i, t}$ and $m_{j, t}$ are both $1$.
This will yield a $1$ to the sum indicative that both vertex $i$ and vertex $j$
are connected to the same edge $t$. The second possible product is a $0$, which
is when either $m_{i, t}$ or $m_{j, t}$ is a $1$ (indicative that one of the
vertices is connected to edge $t$, but not the other) or when both $m_{i, t}$ and
$m_{j, t}$ are $0$, which is indicative that neither vertex is connected to
edge $t$. The resulting sum will be total walks of length $1$ between vertex
$i$ and vertex $j$.

\begin{equation}
    \forall \underset{i\neq j}{i,j} \in[1, q]\quad
    a_{i, j} = \sum_{t = 1}^{r}m_{i, t}m_{j, t}
\end{equation}

This sum is bounded by $r$, or in other words $\sum_{t = 1}^{r}m_{i, t}m_{j, t}
\in [0, r]$.

\begin{equation}
    \forall i \in[1, q]\quad
    d_{V {i, i}} = M_i \cdot M_i = \sum_{t = 1}^{r}m_{i, t}^2
\end{equation}

Let $d_{V {i, i}}$ be the entry at $(i, i)$ in the diagonal matrix $D_V$.
When we take the dot product of the same row vector in the incidence matrix
we get that each entry in the row vector will be squared then summed. Since
each entry is $1$ or $0$, we are effectively taking the sum of all the
entries in the row vector. Entries will only have a $1$ if the edge defined
by its column is incident to the vertex defined by the row, thus the sum
will give us the degree since we are just counting the total number of edges
that are incident to vertex $i$. This will yield the some integer between $0$
and $r$.
\end{document}
